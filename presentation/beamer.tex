\documentclass{beamer}

\usepackage[utf8]{inputenc}
\usepackage{graphicx}
\usepackage{amsmath}
\usepackage{bookmark}
\usepackage[T1]{fontenc}
\usepackage{caption}

\usepackage{colortbl}
\usepackage{xcolor}
\usepackage{tikz}
% \usepackage{enumitem}
\usepackage{amssymb}
\usepackage{makecell}
\usepackage{hyperref}

\title{Projekt: Wykorzystanie CLIP do generowania opisów zdjęć satelitarnych}
\author{Szymon Laszczyński}
\date{\today}

\begin{document}

\frame{\titlepage}

\begin{frame}{CLIP (Contrastive Language-Image Pretraining)}
  Łączenie osadzeń obrazów i tekstów w uwsponionej przestrzeni poprzez uczenie różnicowe (ang. contrastive learning). \\[1em]

  \textbf{Kluczowe cechy:}
  \begin{itemize}
    \item Trenowany na duzych zbiorach \textbf{par obraz-tekst}.
    \item realizuje zadania wizyjne \textbf{zero-shot} bez konieczności dostosowywania modelu.
  \end{itemize}
\end{frame}

\begin{frame}{CLIP - uczenie różnicowe}
  \begin{figure}
    \centering
    \includegraphics[width=0.9\textwidth]{../img/clip-overview-a.png}
    \caption*{\tiny source: https://openai.com/index/clip/}
  \end{figure}
\end{frame}

% \begin{frame}{Przykład}
%   \begin{columns}
%     % Left column for the image
%     \begin{column}{0.45\textwidth}
%       \centering
%       \includegraphics[width=\linewidth]{../img/remote_cats.jpg} % Replace with your image file
%     \end{column}

%     % Right column for the sentences and progress bars
%     \begin{column}{0.55\textwidth}
%       \begin{itemize}[label={}]
%         \item  a photo of a cat
%           \hspace{0.1cm}
%           \begin{tikzpicture}
%             \draw[fill=gray!50] (0,0) rectangle (4,0.1);  % Full bar length
%             \draw[fill=blue] (0,0) rectangle (0,0.1);    % Progress length (60%)
%           \end{tikzpicture}
%           \text{\small 0\%}
%           \vspace{0.2cm}

%         \item  a photo of two tv remotes
%           \hspace{0.1cm}
%           \begin{tikzpicture}
%             \draw[fill=gray!50] (0,0) rectangle (4,0.1);  % Full bar length
%             \draw[fill=blue] (0,0) rectangle (0.32,0.1);    % Progress length (60%)
%           \end{tikzpicture}
%           \text{\small 8\%}
%           \vspace{0.2cm}

%         \item a photo of two cats and two tv remotes
%           \hspace{0.1cm}
%           \begin{tikzpicture}
%             \draw[fill=gray!50] (0,0) rectangle (4,0.1);  % Full bar length
%             \draw[fill=blue] (0,0) rectangle (3.68,0.1);    % Progress length (60%)
%           \end{tikzpicture}
%           \text{\small 92\%}
%           \vspace{0.2cm}

%       \end{itemize}
%     \end{column}
%   \end{columns}
% \end{frame}

\begin{frame}{Cele}
  \textbf{Sukces częściowy:}
  \begin{itemize}
    \item rozpoznawanie obiektów na zdjęciach satelitarnych (zbiór etykiet).
    \item identyfikacja ograniczeń CLIP.
  \end{itemize}
  \textbf{Pełen sukces:}
  \begin{itemize}
    \item generowanie opisu do zdjęć satelitarnych (bez etykiet).
  \end{itemize}
\end{frame}

\begin{frame}{Opis na podstawie etykiet}
  Korzystamy z CLIP by zindentyfikować najbardzie prawdopodobne etykiety ze znanego zbioru.
  \begin{figure}
    \centering
    \includegraphics[width=0.8\textwidth]{../img/clip-zero-shot-from-label.png}
    \caption*{\tiny source: https://openai.com/index/clip/}
  \end{figure}
\end{frame}

\begin{frame}{Wyniki - XView}
  Testowanie na zbiorze z wieloma etykietami. Wszystkich etykiety jest 60. \\[1em]
  Przykłady: \textit{Building, Small Car, Cargo Truck, Utility Truck, Helipad, Damaged Building, Shipping container lot, Construction Site, Facility, Vehicle Lot, Small Aircraft, Storage Tank, Shipping Container}. \\[1em]
  Wybieramy 10 najbardziej prawdopodobnych etykiet.\\[1em]
  \begin{itemize}
    \item 72\% że przynajmniej jedna z etykiet zostala poprawnie zidentyfikowana.
    \item 14\% że wszystkie etykiety zostaly poprawnie zidentyfikowane.
  \end{itemize}
\end{frame}

\begin{frame}{Wyniki - XView}
  \begin{table}[h]
    \centering
    \footnotesize
    \begin{tabular}{cc}
      \includegraphics[width=0.35\textwidth]{../img/xview_1.png} & \includegraphics[width=0.35\textwidth]{../img/xview_2.png}  \\
      \small Bus, Small Car, Building & \small \makecell{Small Car, Utility Truck, \\Cargo Truck, Bus, Building} \\ [1em]
      \makecell[l]{ 30\%  Helipad \\ 14\%  Damaged Building \\ 11\%  Shipping container lot \\ 10\%  \textbf{Building} \\ 7\%   Construction Site \\ 6\%   Facility \\ 4\%   Vehicle Lot \\ 3\%   Small Aircraft } & \makecell[l]{24\%  Shipping container lot \\ 16\%  Helipad \\ 10\%  Vehicle Lot \\ 6\%   Storage Tank \\ 3\%   Shipping Container \\ 3\%   \textbf{Small Car} \\ 3\%   Small Aircraft \\ 2\%   \textbf{Cargo Truck} \\ }  \\
    \end{tabular}
  \end{table}
\end{frame}

\begin{frame}{Wyniki - EuroSAT}
  Testowanie na zbiorze z jedna etykietą dla zdjecia. \\[1em]
  Wszystkich etykiety jest 10: \textit{ Annual Crop, Forest, Herbaceous Vegetation, Highway, Industrial Buildings, Pasture, Permanent Crop, Residential Buildings, River, SeaLake}. \\[1em]
  Wybieramy najbardziej prawdopodobną etykietę.\\[1em]
  \begin{itemize}
    \item 44\% dla oryginalnego zestawu etykiet.
    \item 58\% dla ograniczonego zestawu etykiet.\\
      (field, forest, highway, buildings, body of water)
  \end{itemize}
\end{frame}

\begin{frame}{Wyniki - EuroSAT}
  \begin{table}[h]
    \centering
    \small
    \begin{tabular}{cc}
      \includegraphics[width=0.35\textwidth]{../img/eurosat_1.png} & \includegraphics[width=0.35\textwidth]{../img/eurosat_2.png}  \\
      Pasture &  Highway \\ [1em]
      34\% Pasture & \makecell{42\% Residential Buildings \\ 17\% Highway}  \\ \hline
      75\% field & 45\% highway
    \end{tabular}
  \end{table}
\end{frame}

\begin{frame}{Ograniczenia CLIP}
  \begin{itemize}
    \item Problemy z precyzyjnym identyfikowaniem obiektów
    \item Wrażliwość na dobór etykiet
    \item Bias wywołany przez zbiór treningowy
  \end{itemize}
\end{frame}

\begin{frame}{Opis bez etykiet}
  Trenujemy prosta sieć mapującą osadzenia obrazów na prefix dla prompta gpt-2.
  \begin{figure}
    \centering
    \includegraphics[width=1.0\textwidth]{../img/clipcap-overview.png}
    \caption*{\tiny source: https://arxiv.org/pdf/2111.09734}
  \end{figure}
  Trening przeprowadzamy na zbirze: \texttt{arampacha/rsicd}
\end{frame}

\begin{frame}{Wyniki}
  \begin{figure}
    \centering
    \includegraphics[width=0.8\textwidth]{../img/satellite.png}
  \end{figure}
  several buildings and green trees are near a river in a seaside resort. \\[1em]
  a church is close to a river with many green trees and a pond.
\end{frame}

\begin{frame}{Wyniki}
  \begin{figure}
    \centering
    \includegraphics[width=0.8\textwidth]{../img/tree_in_a_field.png}
  \end{figure}
  Many green trees are in a piece of green meadow.
\end{frame}

\begin{frame}{Wyniki}
  \begin{table}[h]
    \centering
    \small
    \begin{tabular}{cc}
      \includegraphics[width=0.35\textwidth]{../img/xview_2.png} & \includegraphics[width=0.35\textwidth]{../img/eurosat_2.png}  \\
      XView &  EuroSAT \\ [1em]
      \makecell{Many buildings and green \\trees are located in a dense\\ residential area.} & \makecell{Many green trees are\\ planted on both sides of\\ a piece of green meadow.}
    \end{tabular}
  \end{table}
\end{frame}

\begin{frame}{Wyniki}
  \begin{table}[h]
    \centering
    \small
    \begin{tabular}{cc}
      \includegraphics[width=0.35\textwidth]{../img/xview_2.png} & \includegraphics[width=0.35\textwidth]{../img/eurosat_2.png}  \\
      XView &  EuroSAT \\ [1em]
      \makecell{Many buildings and green \\trees are located in a dense\\ residential area.} & \makecell{Many green trees are\\ planted on both sides of\\ a piece of green meadow.}
    \end{tabular}
  \end{table}
\end{frame}

\begin{frame}{Repozytorium i żródła}
  projekt: \url{https://github.com/simon-joseph/uwr-llm-project} \\
  żródła:
  \begin{itemize}
    \item CLIP : \url{https://openai.com/index/clip/}
    \item CLIPCAP : \url{https://arxiv.org/pdf/2111.09734}
    \item CLIPCAP : \url{https://github.com/rmokady/CLIP_prefix_caption}
  \end{itemize}
  zbiory danych:
  \begin{itemize}
    \item \url{huggingface.co/datasets/CHichTala/xview}
    \item \url{huggingface.co/datasets/blanchon/EuroSAT_RGB}
    \item \url{huggingface.co/datasets/arampacha/rsicd}
  \end{itemize}
\end{frame}

\end{document}
