\documentclass{beamer}

\usepackage[utf8]{inputenc}
\usepackage{graphicx}
\usepackage{amsmath}
\usepackage{bookmark}
\usepackage[T1]{fontenc}

\title{Projekt: Wykorzystanie CLIP do rozpoznawania obiektów na zdjęciach satelitarnych}
\author{Szymon Laszczyński}
\date{\today}

\begin{document}

\frame{\titlepage}

\section{Czym jest CLIP?}
\begin{frame}{Na czym polega CLIP?}
  \begin{itemize}
    \item CLIP (Contrastive Language-Image Pretraining): Model stworzony przez OpenAI do łączenia obrazów i tekstów w jednej przestrzeni zanurzeń.
    \item Kluczowe cechy:
      \begin{itemize}
        \item Trening na dużych, różnorodnych zbiorach danych (obraz+tekst).
        \item Uniwersalność zastosowań: od analizy obrazów po wyszukiwanie informacji.
      \end{itemize}
  \end{itemize}
\end{frame}

\begin{frame}{CLIP: Przykład 1}
  \begin{columns}
    \begin{column}{0.5\textwidth}
      \begin{figure}
        \centering
        \includegraphics[width=0.8\textwidth]{../../img/remote_cats.jpg}
      \end{figure}
      \begin{itemize}
        \item a photo of a cat: 0.99
        \item a photo of a dog: 0.01
        \item --
        \item a photo of a cat: 0.91
        \item a photo of a tv remote: 0.09
      \end{itemize}
    \end{column}
    \begin{column}{0.5\textwidth}
      \begin{itemize}
        \item a photo of a cat: 0.01
        \item a photo of two tv remotes: 0.99
        \item --
        \item a photo of a cat: 0.00
        \item a photo of two tv remotes: 0.08
        \item a photo of two cats and two tv remotes: 0.92
      \end{itemize}
    \end{column}
  \end{columns}
\end{frame}

\begin{frame}{CLIP: Przykład 2}
  \begin{columns}
    \begin{column}{0.5\textwidth}
      \begin{figure}
        \centering
        \includegraphics[width=0.8\textwidth]{../../img/satellite.png}
      \end{figure}
      \begin{itemize}
        \item 0.05 : a photo of a car
        \item 0.40 : a photo of a river
        \item 0.56 : a photo of a city
      \end{itemize}
    \end{column}
    \begin{column}{0.5\textwidth}
      \begin{itemize}
        \item 0.00 : a photo of a car
        \item 0.04 : a photo of a river
        \item 0.06 : a photo of a city
        \item 0.89 : a photo containing a part of a city with a river, buildings, roads and some cars in the parking lots
      \end{itemize}
    \end{column}
  \end{columns}
\end{frame}

\begin{frame}{CLIP: Przykład 2 : uwaga!}
  \begin{itemize}
    \item 0.00 : a photo of a car
    \item 0.01 : a photo of a river
    \item 0.01 : a photo of a city
    \item 0.11 : a photo containing a part of a city with a river, buildings, roads and some cars in the parking lots
    \item \textcolor{red}{0.88 : a satellite image of a seacoast with industrial buildings and ship containers}
  \end{itemize}
\end{frame}

\section{Opis projektu}
\begin{frame}{Opis}
  \begin{itemize}
    \item Cel: Wykorzystanie modelu CLIP do rozpoznawania obiektów na zdjęciach satelitarnych.
    \item Rozszerzenie: Analiza fragmentów muzyki w celu rozpoznawania gatunku muzycznego.
  \end{itemize}
\end{frame}

\section{Charakterystyka projektu}
\begin{frame}{Ogólna charakterystyka}
  \begin{itemize}
    \item Przeliczenie zanurzeń dla polskiego LLM i encodera dla obrazów.
    \item Dotrenowanie dla specjalistycznych danych (zdjęcia satelitarne).
    \item Eksperymenty z użyciem modelu:
      \begin{itemize}
        \item Rozpoznawanie obiektów na zdjęciach satelitarnych.
        \item Oznaczanie obszarów z obiektami.
        \item Zliczanie obiektów.
      \end{itemize}
    \item Badanie możliwości przeniesienia metod na inne dziedziny (np. muzykę).
  \end{itemize}
\end{frame}

\section{Narzędzia i dane}
\begin{frame}{Dodatkowe narzędzia i dane}
  \begin{itemize}
    \item Narzędzia:
      \begin{itemize}
        \item huggingface
        \item OpenAI CLIP: github.com/openai/CLIP
        \item Narzędzia do przetwarzania danych geograficznych (np. GDAL, QGIS).
      \end{itemize}
    \item Dane:
      \begin{itemize}
        \item EuroSAT github.com/phelber/eurosat
        \item Sentinel-2 bigearth.net
        \item github.com/Yuan-ManX/ai-audio-datasets
      \end{itemize}
  \end{itemize}
\end{frame}

\section{Definicje sukcesu}
\begin{frame}{Definicje sukcesu projektu}
  \begin{itemize}
    \item Pełen sukces:
      \begin{itemize}
        \item Opracowanie rozszerzenia do analizy muzyki, które działa w praktyce.
        \item Model poprawnie generuje annotacje do zdjęć satelitarnych.
      \end{itemize}
    \item Sukces częściowy:
      \begin{itemize}
        \item Model jest w stanie rozpoznawać obiekty na zdjęciach satelitarnych.
        \item Identyfikacja ograniczeń CLIP w specyficznych zadaniach.
      \end{itemize}
  \end{itemize}
\end{frame}

\end{document}
